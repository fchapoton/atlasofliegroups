\documentclass[11 pt]{article}

\addtolength{\textwidth}{2 cm}
\addtolength{\hoffset}{-1 cm}
\addtolength{\textheight}{2 cm}
\addtolength{\voffset}{-1 cm}

\usepackage{amsmath}
\usepackage{amssymb}

\def\ds{\displaystyle}
\def\ov{\overline}
\def\ra{\rightarrow}
\def\rao#1{\overset{#1}\ra}

\def\1{^{-1}}
\def\a{\alpha}
\def\ac{\alpha^\vee}
\def\af{{\mathfrak a}}
\def\b{{\mathfrak b}}
\def\C{{\bf C}}
\def\D{{\cal D}}
\def\d{\delta}
\def\e{\varepsilon}
\def\F{\Phi}
\def\G{{\bf G}}
\def\GL{{\bf GL}}
\def\g{\gamma}
\def\gf{{\mathfrak g}}
\def\H{{\bf H}}
\def\HC{{\cal HC}}
\def\Hc{{\cal H}}
\def\Hom{{\rm Hom}}
\def\hf{{\mathfrak h}}
\def\K{{\bf K}}
\def\Ker{{\rm Ker}}
\def\kf{{\mathfrak k}}
\def\L{\Lambda}
\def\l{\lambda}
\def\lf{{\mathfrak l}}
\def\M{{\cal M}}
\def\mf{{\mathfrak m}}
\def\N{{\bf N}}
\def\pf{{\mathfrak p}}
\def\Q{{\bf Q}}
\def\qh{q^{1/2}}
\def\qmh{q^{-1/2}}
\def\R{{\bf R}}
\def\t{\theta}
\def\tf{{\mathfrak t}}
\def\Z{{\bf Z}}
\def\Zq{\Z[q^{1/2},q^{-1/2}]}

\begin{document}

\begin{center}
\textbf{Computing Kazhdan-Lusztig polynomials for real Lie groups}\\
\bigskip
\textit{Fokko du Cloux}\\
\textit{Institut Girard Desargues, UMR CNRS 5028}\\
\textit{Universit\'e Lyon-I}\\
\textit{F--69622 Lyon Cedex, FRANCE}
\end{center}

\bigskip

\begin{quote}
\noindent Informal notes for the workshop of the
Atlas of Real Lie Groups project, held at the American Institute of
Mathematics, Palo Alto, CA, in July 2003. Needless to say, all this
is lifted from David Vogan's work.
\end{quote}

\bigskip

\noindent\textbf{1. Setup}

\medskip

\noindent Let $\G$ be a connected reductive complex algebraic group, defined
over $\R$, and let $G$ be a subgroup of $\G$ such that $\G(\R)^\circ\subset G
\subset\G(\R)$. Let $\t$ be a Cartan involution of $G$, and let $K=G^\t$ be
the corresponding maximal compact subgroup of $G$. Using lowercase German
letters for the corresponding Lie algebras, we have a decomposition
$$
\gf=\kf\oplus\pf
$$
where $\kf$ ($\pf$) is the $+1$ ($-1$) eigenspace of $\t$ on $\gf$.

Let $Z(\gf)$ be the center of the universal enveloping algebra of $\gf$, and
fix a character $\chi$ of $Z(\gf)$. Let $\HC$ be the category of Harish-Chandra
modules for $G$ (i.e.\ finitely generated $(\gf,K)$--modules), and let 
$\HC_\chi$ be the full subcategory of $\HC$ of the
modules with generalized central character $\chi$. We are interested in the
distribution characters, say, of the irreducible modules in $\HC_\chi$.

These are of course a basis of the Grothendieck group of $\HC_\chi$. Now there
is another basis $\D$ of this Grothendieck group, which we may consider known,
made up by the characters of standard representations (parabolically induced
from discrete series on cuspidal parabolic subgroups.) To each $\g\in\D$,
Langlands assigns a specific irreducible subquotient $\ov\g$; this establishes
a bijection from $\D$ to the irreducibles in $\HC_\chi$. If we denote
$m(\ov\g,\d)$ the multiplicity of $\ov\g$ in $\d$, we can write
$$
\d=\sum_{\g\in\D}m(\ov\g,\d)\ov\g
$$
in the Grothendieck group (or as distribution characters). In a suitable
ordering of $\D$, the matrix $m(\ov\g,\d)$ is unipotent upper triangular; so
we can invert it and write
$$
\ov\d=\sum_{\g\in\D}M(\g,\ov\d)\g
$$
The problem is to compute the integers $M(\g,\ov\d)$.

\medskip

\begin{quote}\noindent
In the sequel, we will assume for simplicity that $\chi$ is the 
infinitesimal character of the trivial representation of $G$. Also we fix a 
Cartan subalgebra $\hf^a$ of $\gf$, and a Borel
subalgebra $\b^a$ containing $\hf^a$.
Denote $W=W(\gf,\hf^a)$.
\end{quote}

\vfill\eject

\noindent\textbf{2. Kazhdan-Lusztig polynomials}

\medskip

Let's take the problem to a higher level (to understand where this comes from,
one needs the geometric interpretation of the category $\HC_\l$ in terms of 
$\K$--equivariant constructible sheaves on the flag manifold of $\G$, where 
$\K$ is the complexification of $K$.) Let $\M$ be the free $\Zq$--module with
basis $\D$.

Let $\Hc=\Hc(W)$ be the Hecke algebra of $W$; it is an algebra over the
same ring $\Zq$. The main ingredient in all that follows will be the
definition of an $\Hc$--module structure on $\M$, which we shall describe
later; also we shall define a certain graded partial order relation $\leq$ on 
$\D$, with associated length function $l$. It will turn out that there is a 
unique $\Zq$--antilinear involution $D$ on $\M$ such that

\begin{itemize}
\item[\it(a)]$D(hm)=D(h)D(m)$ for all $h\in\Hc$, $m\in\M$, where we denote
also by $D$ the usual Kazhdan--Lusztig involution on $\Hc$.
\item[\it(b)]$D(t_\d)=t_\d+\sum_{\g<\d}r(\g,\d)t_\g$, $r(\g\d)\in\Zq$, where 
we have put $t_\g=q^{-l(\g)/2}\g$ for each $\g\in\D$.
\end{itemize}

Then, from general principles, there is a unique basis $c_\d$ of $\M$ (the 
Kazhdan-Lusztig basis), such that

\begin{itemize}
\item[\it(a)]$\ds{c_\d=t_\d+\sum_{\g<\d}p(\g,\d)t_\g}$\hspace{.5cm}
$p(\g,\d)\in q^{-1/2}\Z[q^{-1/2}]$
\item[\it(b)]$D(c_\d)=c_\d$
\end{itemize}

It will turn out that $P_{\g,\d}=q^{(l(\d)-l(\g))/2}p(\g,\d)$ belongs to
$\Z[q]$, and that its evaluation at $q=1$ yields the $M(\g,\ov\d)$ up to sign~:
$$
M(\g,\ov\d)=(-1)^{l(\d)-l(\g)}P_{\g,\d}(1)
$$

The computation of the basis $c_\d$ is a purely formal exercise once the
involution $D$ is known. In practice, however, making use of the $\Hc$--action
we can get direct recursion formulas which are probably quite a bit more 
efficient than the formal approach.

The $P_{\g,\d}$ above are the \textit{Kazhdan-Lusztig polynomials} for $G$.

\bigskip

\noindent\textbf{3. Parametrization of the irreducible representations}

\medskip

If we want to do anything at all explicitly of course our first task should be
to make the set $\D$ explicit. Here is how this goes. The set $\D$ is the
set of $K$--conjugacy classes of triples $(H,\l,\F)$, where $H$ is a 
$\t$--stable Cartan subgroup of $G$, $\l\in\hf^*$ is in the $W(\gf,\hf)$--orbit
corresponding to $\chi$ under the Harish-Chandra isomorphism 
$Z(\gf)\ra S(\hf)^W$, and $\F$ is
a character of the compact part $T$ of $H$, whose differential $d\F$ is
known from $\l$. The precise definition is this~: let $H=TA$ be the Cartan
decomposition of $H$, and let $L$ be the centralizer of $A$ in $G$. Then
we have a decomposition $L=MA$, and $\tf$ is a Cartan subalgebra of $\mf$.
Since $\l$ is a regular weight, no coroot vanishes on $\l$. Let
$$
\rho^I_\l=\frac{1}{2}\sum_{\langle\ac,\l\rangle>0\atop\a\in\Delta(\mf,\tf)}\a
\qquad\qquad
\rho^{I,c}_\l=\frac{1}{2}\sum_{\langle\ac,\l\rangle>0\atop\a\in
\Delta(\kf\cap\mf,\tf)}\a
$$
be the half-sums of roots of $\tf$ in $\mf$ (resp. $\kf\cap\mf$) which are
positive w.r.t. $\l$. Then we ask that
$$
d\F=\l|_\tf+\rho^I_\l-2\rho^{I,c}_\l
$$
Since the component group of $T$ is a commutative $2$--group, there are as many
such $\F$ as there are connected components in $T$, and this number is a power
of $2$.

The $K$--conjugacy classes of $\t$--stable Cartan subgroups are in $(1,1)$
correspondence with the $G$--conjugacy classes of Cartan subgroups in $G$;
there are finitely many. If we choose a set of representatives 
$\{H_i\}_{i\in I}$ of $\t$--stable Cartans, we see that for each $i\in I$,
$(\l,\F)$ has to be considered up to the action of the normalizer of $H_i$
in $K$, divided by the centralizer; this is also the Weyl group $W(G,H_i)$.

Let $\L_i\subset\hf_i^*$ be the $W(\gf,\hf_i)$--orbit corresponding to our 
chosen infinitesimal character $\chi$. Then for each $\l\in\L_i$ there is a 
unique isomorphism $i_\l:\hf^a\ra\hf_i$ which takes the positive chamber
defined by $\b^a$ to the positive Weyl chamber in $\hf_i$ defined by $\l$,
and which is induced by conjugation with an element of ${\rm Ad}(\gf)$. So
$\L_i$ is in canonical bijection with a subset of $\Hom(\hf^a,\hf_i)$, and
we have a canonical right action of $W$ on $\L_i$ by composition on the right;
this commutes with the left action of $W(\gf,\hf_i)$ and in particular with
the left action of $W(G,H_i)$. 

Let $\D_i$ be the part of $\D$ corresponding to $H_i$. The upshot is that we 
have a decomposition
$$
\D=\coprod_{i\in I}\D_i
$$
where each $\D_i$ is fibered over a the set of $W(G,H_i)$--orbits
in $\L_i$, and each fiber is provided with a simply transitive action of the 
group $(H_i/H_i^\circ)^{\wedge}$.

\bigskip

\noindent\textbf{4. The cross action and the Cayley transforms}

\medskip

The Hecke algebra action on $\M$ lifts the coherent 
continuation action of the group $W$ on the Grothendieck group of $\HC_\l$. 
It is convenient to first introduce an action of $W$ on the set $\D$, the 
so-called the \textit{cross-action}. For this action, the group $W$ acts in 
fact separately on each $\D_i$.

The action is just a lift of the action of $W$ on $\L_i$, defined in the 
previous section. To make it a left action, set $w\times\l=\l w\1$. We may
also write this as $w\times\l=w_\l\1(\l)$, with $w_\l=i_\l wi_\l\1\in 
W(\gf,\hf_i)$. Clearly $w\times\l-\l$ is an integral linear combination of
roots. Also, $\rho^I_\l-\rho^I_{w\times\l}$ and 
$\rho^{I,c}_\l-\rho^{I,c}_{w\times\l}$ are integral linear combinations of
roots in $\Delta(\mf_i,\tf_i)$. Now it turns out that the differential is
a bijection from the lattice in the space $X(H_i)$ of characters of $H_i$ 
generated by the root characters, to the root lattice in $\hf_i$. So there is 
a unique element in the ``root lattice'' of $X(H_i)$ corresponding to
$$
(w\times\l+\rho^I_{w\times\l}-2\rho^{I,c}_{w\times\l})
-(\l+\rho^I_\l-2\rho^{I,c}_\l)
$$
This element we take as $w\times\F-\F$. Now if $\g=(H_i,\l,\F)$, we set
$w\times\g=(H_i,w\times\l,w\times\F)$. Of course it should be checked that 
this is indeed an action.

\medskip

As a further preparation for the definition of the Hecke algebra action on 
$\M$, we need the so-called Cayley transforms. These will establish a link 
between representations associated to various Cartans.

Let $\a\in\Delta(\gf,\hf^a)$ be a simple root; and fix $\g=(H_i,\l,\F)\in\D$.
Set $\a_\l=i_\l(\a)\in\Delta(\gf,\hf_i)$. Then the Cayley transform $c^\a(\g)$
is defined whenever $\a_\l$ is noncompact imaginary; the idea is that it will
take $\g$ to an element associated with a Cartan where $\a$ is {\em real}. Let
$\hf_i=\tf_i\oplus\af_i$ be the Cartan decomposition of $\hf_i$. Let $\lf$ be
the subalgebra of $\gf$ generated by $\hf_i$ and the root vectors $\pm\a_\l$.
Then $\lf$ is $\t$--stable, defined over $\R$, and has exactly two $\t$--stable
conjugacy classes of Cartan subalgebras; $\hf_i$ represents the maximally
compact one, and the other, say conjugate to $\hf_j$ in $\gf$, the maximally
split one. In fact, $\hf_i\cap\hf_j$ is of codimension one in each of them
if we take the representative of $\hf_j$ to lie inside $\lf$. Then there are
just two $\l'\in\L_j$ such that $\l'|_{\Ker\a}=\l|_{\Ker\a}$; moreover these
two $\l'$ are conjugate under the reflection by $\a$, which now lies in
$W(G,H_j)$. So $(H_j,\l')$ is well-defined up to $K$--conjugacy.

It remains to see what happens with the $\F$--part of $\g$. What we want is
to consider all $\g'=(H_j,\l',\F')$ such that $\F=\F'$ on $T_i\cap T_j$. There
are
two cases : (a) $T_j\subset T_i$, in which case $\F'$ is entirely determined,
and we set $c^\a(\g)=\{\g'\}$ (b) $T_j\not\subset T_i$; then $T_i\cap T_j$ is
of index two in $T_j$, and it is easy to see that in this case 
$T_j\simeq(T_i\cap T_j)\times\Z/2\Z$; hence there are exactly two possibilities
for $\F'$, traditionally labelled $\g^\a_+$ and $\g^\a_-$, although this
notation is a bit unfortunate because really the two are indistinguishable
at this point. We say that $\a$ is {\em type I} in the first case, {\em type 
II} in the second case.

The inverse Cayley transform goes in the other direction, from the more
split Cartan to the more compact one. It turns out that there is a nice
duality~: when $\a$ is type one for $\g$, then it is also for $s\times\g$,
we have $s\times\g\neq\g$,
$c^\a(s\times\g)=c^\a(\g)$, and those are the only two elements $\d\in\D$
such that $\g^\a\in c^\a(\d)$; we will write $c_\a(\g^\a)=\{\g,s\times\g\}$.
When $\a$ is type II for $\g$, we have $s\times\g=\g$, 
$s\times\g^\a_+=\g^\a_-$, $s\times\g^\a_-=\g^\a_+$, and $\g$ is the only
element $\d\in\D$ such that $c^\a(\d)\cap\{\g^\a_+,\g^\a_-\}$ is non-empty;
we write $c_\a(\g^\a_+)=c_\a(\g^\a_-)=\g$. In short, the domain of $c_\a$ is
the range of $c^\a$, and the image of $\g$ under $c_\a$ is its preimage
under $c^\a$; when $c_\a(\g)$ has two elements, we also denote it as
$\{\g^+_\a,\g^-_\a\}$. An element $\g\in\D$ can be in the domain of $c_\a$
only if $\a_\l$ is real; if it is in the domain, we say that $\a$ is type I
real if it comes from a type I noncompact imaginary root, so that
$c_\a(\g)$ has two elements; and type II real if it comes from a type II
noncompact imaginary root, so that $c_\a(\g)$ has one element.

\bigskip

\noindent\textbf{5. The Hecke algebra action and the Bruhat order}

\medskip

Now we are ready for the definition of the Hecke operators. Given $\g\in\D$ 
and a Coxeter generator $s\in W$, corresponding to a simple root 
$\a\in\Delta(\gf,\hf^a)$, we define $T_s\g\in \M$ as follows.

\begin{itemize}
\item[\it(a)]if $\a_\l$ is compact imaginary, then $T_s\g=q\g$;
\item[\it(b)]if $\a_\l$ is noncompact type I imaginary, then 
$T_s\g=s\times\g+\g^\a$;
\item[\it(c)]if $\a_\l$ is noncompact type II imaginary, then
$T_s\g=\g+\g^\a_++\g^\a_-$;
\item[\it(d)]if $\a_\l$ is complex, and $\t_\l\a\in\Delta^+(\gf,\hf^a)$,
then $T_s\g=s\times\g$;
\item[\it(e)]if $\a_\l$ is complex, and $\t_\l\a\in\Delta^-(\gf,\hf^a)$,
then $T_s\g=q\>s\times\g+(q-1)\g$;
\item[\it(f)]if $\a$ is type I real for $\g$, then
$T_s\g=(q-2)\g+(q-1)(\g_\a^++\g_\a^-)$;
\item[\it(g)]if $\a$ is type II real for $\g$, then
$T_s\g=(q-1)\g-s\times\g+(q-1)\g_\a$;
\item[\it(h)]if $\a$ is real and $\g$ is not in the domain of $c_\a$,
then $T_s\g=-\g$.
\end{itemize}

\medskip

Also, we can now define the \textit{length function} on $\D$. For each
$\g=(H_i,\l\F)\in\D$, transfer the involution $\t$ on $\hf_i$ to $\hf^a$
by setting $\t_\l=i_\l\1\t i_\l$. Then $\t_\l$ is an involution of $\hf^a$
which preserves the root system; in particular it takes Weyl chambers to
Weyl chambers. So we may define
$$
L(\g)=\frac{1}{2}\#\left\{\a\in\Delta^+(\gf,\hf^a)\mid\t_\l(\a)\in
\Delta^-(\gf,\hf^a)\right\}+\frac{1}{2}\dim(\textrm{$-1$ eigenspace of 
$\t_\l$})
$$
This is a half-integer, but length-differences between comparable elements
in the Bruhat ordering (which we are about to define) are always integers.

\medskip

Let $s\in S$ be given, and $\g,\g'\in\D$. We write $\g\rao s\g'$ if 
$L(\g)=L(\g')+1$, and $\g'$ appears in $T_s\g$. It is equivalent to say
that either (a) $\a_\l$ is complex, $\t_\l(\a)\not\in\Delta^+(\gf,\hf^a)$, 
and $\g'=s\times\g$, or (b) $\a_\l$ is real, $\g$ is in the domain of $c_\a$,
and $\g'$ occurs in $c_\a(\g)$. For a given $\g$, the set of $s$ for which
there is an arrow $\g\rao s\g'$ is denoted by $\tau$, and called the
abstract $\tau$--invariant (or the descent set) of $\g$.

The {\em Bruhat ordering} on $\D$ is defined as follows. It is the weakest
partial order $\leq$ with the property that for each $\g,\g'\in\D$ and $s\in S$
such that $\g\rao s\g'$ we have $\d\leq\g$ if and only if either $\d\leq\g'$,
or there are $\d,\d'\in\D$ such that $\d'\leq\g'$ and $\d\rao s\d'$. Note that
contrary to the group case, it is necessary to take into account {\em all} the
$s\in\tau(\g)$, and not justo one of them.

\medskip

This completes the definition of all the data occurring in the Kazhdan-Lusztig
algorithm.

\bigskip

\noindent\textbf{6. Recursion formulas}

\medskip

As we have mentioned in sect. 2, there are two possible approaches for the
computation of the Kazhdan-Lusztig polynomials. Either one determines first
the involution $D$, and deduces the $c_\g$ purely formally, or one sets up
direct recursion formulas for the $c_\g$. In practice, the second approach
is more efficient; the difficulties in making the recursion effective are the 
same anyway, whether one wants to determine $D$ or the $c_\g$.

\medskip

Let us review how things work out for the Hecke algebra. It is better to
replace the basis $T_w$ by $t_w=q^{-l(w)/2}T_w$. The involution on the
Hecke algebra is an antilinear algebra involution, given on $t_w$ by
$D(t_w)=t_{w\1}\1$. It is easy to see that $t_s\1=t_s+\a$, where
$\a=q^{-1/2}-q^{1/2}$, and that the element $c_s$ of the Kazhdan-Lusztig
basis of $\Hc$ is given by $c_s=t_s+\qmh$.

Then to compute the $c_w$ by induction on the length of $w$, we assume
that $l(w)>0$, we pick $s\in S$ such that $sw<w$, and we look at 
$c_sc_{sw}\in\Hc$. This is certainly a selfdual element, of the form
$$
c_sc_{sw}=t_w+\sum_{z<w}a_zt_z
$$
It is easy to see that $a_z\in\Z[\qmh]$; they might, however, have constant
terms. In fact, it is easy to see that a constant term occurs in $a_z$ if
and only if $z<ws$, $sz<z$, and the element $p(z,ws)$ has a term in degree
$-1/2$. If we denote the coefficient of degree $-1/2$ in $p(x,y)$ by 
$\mu(x,y)$, we then see that
$$
c_sc_{sw}-\sum_{z<sw\atop sz<z}\mu(z,sw)c_z
$$
satisfies all the conditions required of $c_w$. When we write the corresponding
formulas for the coefficients of $c_w$, we get the familiar recursion formulas
for Kazhdan-Lusztig polynomials.

\medskip

In the current situation, one could try something similar. Let us denote $l$
the absolute length function on $\D$, {\em i.e.} the one where minimal elements
have length $0$, and for non-minimal $\g$, the length of $\g$ is one more than
the length of the maximal elements $<\g$. Define $t_\g$ as in section 2.

Clearly for $\g$ minimal, we have $c_\g=t_\g$. Now if $\g$ is not minimal,
there is at least one $\g'$, and an element $s\in S$, such that $\g\rao s\g'$.
Then we may consider the element $c_sc_{\g'}\in \M$. Again, it is clearly 
selfdual. Let $\a\in\Delta(\gf,\hf^a)$ be the simple root corresponding to
$s$. Now there are two cases~:

\begin{itemize}
\item[\it(a)]$\a_{\l'}$ is complex, or noncompact type I imaginary. Then 
$c_sc_{\g'}$ is of the form
$$
t_\g+\sum_{\d<\g}a_\d t_\d
$$
and we have $a_\d\in\Z[\qmh]$ just as in the Hecke algebra case. Defining
$\mu(\d,\g)$ exactly in the same way, we may subtract an integral linear
combination of $c_\d$ for $\d<\g$, to obtain the required basis element $c_\g$.
\item[\it(b)]$\a_{\l'}$ is noncompact type II imaginary. Then $c_sc_{\g'}$ is
of the form~:
$$
t_\g+t_{s\times\g}+\sum_{\d<\g}a_\d t_\d
$$
and the same kind of thing will lead to a formula for $c_\g+c_{s\times\g}$.
\end{itemize}

It is not so easy to get around the difficulty in case {\em (b)}. Denote $\sim$
the smallest equivalence relation on $\D$ such that $\g$ and
$s\times\g$ are equivalent whenever $\a_\l$ is type II real. Clearly the
formulas in (b) will allow to compute the $c_\g$ for all $\g$ in a given
equivalence class, once one of them is known. So we are fine if at least one
$\d\sim\g$ is such that $\tau(\d)$ is not made un entirely of type II real
roots. Notice also that if there is a loop in the graph defined on the class
by the edges $\g,s\times\g$, $s\in\tau(\g)$, where are again fine. So the
case where we seem to be really stuck is when in addition this graph is a tree;
the simplest example of this arises in $\GL(2,\R)$ already.

Nevertheless it may be shown that there is always a way out, which will allow
the computation of $p(\d,\g)$ using induction on $l(\g)-l(\d)$ --- I'll not go
into this as it is a little bit technical and I don't feel I have understood
it well enough yet.

\bigskip

\noindent\textbf{7. Computational assessment}

\medskip

It seems to me that the difficulties in the computation lie mostly within
the realm of structure theory. I think one should take everything back to our 
fixed Cartan algebra $\hf^a$. Then at least the set of $\l$ corresponding
to the central character $\chi$ will be fixed.

In fact, it is probably best to work within a fundamental Cartan subgroup
$H=TA$. It is known that conjugacy classes of Cartan subgroups are in
$(1,1)$ correspondence with $W(K,T)$--conjugacy classes of sets of
strongly orthogonal noncompact imaginary roots.

To each class $H_i$ will correspond to an involution $\t_i$ on $\hf^a$.
Determining the component group of $H_i$ from $\t_i$ is not hard.
One of the delicate things will be to determine the Weyl group $W(G,H_i)$,
but there are algorithms for that in Vogan's papers. 
Another delicate aspect is how to describe the component
group of $\G(\R)$, and how to keep track of the difference between $G$ and
$\G(\R)$ in the computations. In fact, it might be reasonable as a first
approach to assume $G=\G(\R)$.

The cross-action of $W$ should be no problem, although it involves the
determination of the action of $W$ on $W/W(G,H_i)$, which is not so easy,
as $W(G,H_i)$ is not a reflection subgroup in general. This will probably
require some version of Todd-Coxeter coset enumeration.

What looks a bit more delicate
is the Cayley transform, but perhaps not that much : after all, the new $\t_j$
should basically correspond to changing $\a$ from a $+1$--eigenvector into
a $-1$--eigenvector, and the new $+1$--eigenspace is just $\Ker(\a)$. (check
this!)

Once all these are defined, we should be in good shape. The only thing left,
then, would be a reasonably efficient implementation of the recursion process;
this may involve some fairly complicated parts for those rare cases where
one gets stuck in the situation described in sect. 6, but still should not
be all that different from the Hecke algebra case.

\bigskip

\noindent\textbf{References}

\begin{itemize}\itemsep0 pt
\item[{[1]}]G. Lusztig and D.A. Vogan, Jr, Singularities of Closures of 
$K$--orbits on Flag Manifolds, {\it Invent. math.} {\bf 71}(1983), pp. 
365--379.
\item[{[2]}]D.A. Vogan, Jr, {\it Representations of Real Reductive Lie Groups},
Birkh\"auser, Boston, 1981.
\item[{[3]}]D.A. Vogan, Jr, The Kazhdan--Lusztig conjecture for real reductive
Lie groups, (Park City conference in 1982, Birkhauser, get correct reference!)
\item[{[4]}]D.A. Vogan, Jr, Irreducible characters of Semisimple Lie Groups 
III. Proof of the Kazhdan--Lusztig Conjecture in the Integral Case, {\it
Invent. math.} {\bf 71}(1983), pp. 381--417.
\end{itemize}

\end{document}