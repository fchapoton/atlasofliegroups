\documentclass[11pt,draft]{article}

\makeatletter
\renewcommand{\subsection}[1]{\@startsection{subsection}{2}{0pt}%
{\medskipamount}{-4.5 pt}{\textbf}{#1}\hskip-\parindent\textbf{. }}
\makeatother

\usepackage{amsmath}
\usepackage{amssymb}

\newcommand{\ra}{\rightarrow}

\newcommand{\A}{{\bf A}}
\renewcommand{\a}{\alpha}
\newcommand{\C}{{\bf C}}
\newcommand{\D}{\Delta}
\newcommand{\F}{{\bf F}}
\newcommand{\f}{{\varphi}}
\newcommand{\G}{{\bf G}}
\newcommand{\GL}{{\bf GL}}
\newcommand{\g}{{\mathfrak g}}
\renewcommand{\H}{{\bf H}}
\newcommand{\h}{{\mathfrak h}}
\renewcommand{\Im}{{\rm Im}}
\newcommand{\K}{{\bf K}}
\newcommand{\Ker}{{\rm Ker}}
\renewcommand{\k}{{\mathfrak k}}
\renewcommand{\l}{\lambda}
\newcommand{\p}{{\mathfrak p}}
\newcommand{\Q}{{\bf Q}}
\newcommand{\R}{{\bf R}}
\renewcommand{\S}{{\bf S}}
\newcommand{\SL}{{\bf SL}}
\newcommand{\T}{{\bf T}}
\newcommand{\U}{{\bf U}}
\newcommand{\Z}{{\bf Z}}

\begin{document}

\title{Algorithms for structure theory}
\author{Fokko du Cloux}
 
\maketitle

These are draft notes for the first part of the Atlas of reductive Lie groups
project, dealing with the structure theory part of the program. Computationaly,
most of this should be very light, and certainly much or all of it could be
implemented in a high-level environment such as {\tt Maple} or {\tt GAP}.
However, since the structure theory things lie at the very heart of the
program, it is necessary to have them available at the C++ level as well.

\section{Defining the group}\label{section:group}

\subsection{}Let $\G$ be a connected complex reductive Lie group, defined
over $\R$; let $\G(\R)$ be the full group of real points, $\G(\R)^\circ$ its
identity component. We will be interested in Lie groups $G$ such that
$\G(\R)^\circ\subset G\subset\G(\R)$; such groups we call {\em real reductive
Lie groups} in these notes. Note that later on one might wish to consider
finite (or even infinite ?) covers of such groups $G$, but for now we refrain
from that. Denote $\g$ the Lie algebra of $\G$, also the complexified Lie
algebra of $G$.

\subsection{}The datum of $\G$ is equivalent to that of a {\em root datum}~:
this is a quadruple $(X,R,X^\vee,R^\vee)$ where $X$ and $X^\vee$ are two
lattices in duality, and $R\subset X$ and $R^\vee\subset X^\vee$ are two
finite subsets, for which we are given a bijection denoted $\a\ra\a^\vee$ such
that {\em (a)} $\langle\a,\a^\vee\rangle=2$ for all $\a\in\R$, and {\em (b)}
for each $\a\in R$ the reflection $s_\a:\l\ra\l-\langle\l,\a^\vee\rangle\,\a$
takes $R$ into $R$, and similarly $s_{\a^\vee}$ takes $R^\vee$ into $R^\vee$.
In particular, this means that $R$ is a root system within the real vector
space it spans in $X\otimes_\Z\R$.

To set up the root datum, choose a maximal torus $\T$ in $\G$, and let
$X$ be the character group of $T$, $X^\vee$ the cocharacter group ({\em i.e.},
the group of homomorphisms from $\C^\times$ to $T$), let $R$ be the set of 
roots of $\T$ in $\g$, $R^\vee$ the set of coroots. We will also say that
$X$ is the {\em weight lattice} of $\G$.

\subsection{}\label{subsection:lie type}{\em Entering a group into the 
system~: Lie type.} Entering a group into the system requires quite a 
bit of user interaction. The group $\G$ always
has a finite cover $\tilde\G$ which is the direct product of a simply
connected complex semisimple group $\tilde\G'$ and a torus $R(\tilde\G)$.
Taking into account the
real structure, we may split up $\tilde\G'$ into a product of groups of the
following forms (where ``simple'' really means ``quasi-simple'')~: 
{\em (a)} simply connected simple groups defined
over $\R$ {\em (b)} simply connected simple complex groups, viewed as real
groups (such a group $G_1$ is a real form of $G_1\times G_1$). Similarly,
the radical $R(\tilde\G)$ is a product of tori defined over $\R$ which are
either anisotropic (or, as we prefer to say, compact, {\em i.e.}, such that 
$\T(\R)\simeq\U(1)^n$ is compact), split ($T(\R)\simeq\R^\times{}^n$), or 
complex ($T(\R)\simeq\C^\times{}^n$.)

The user enters a sequence of symbols specifying the factors of $\tilde\G$,
and for each factor specifies a real form (complex factors actually use up
two consecutive isomorphic factors of $\tilde\G$.) From this, the program
knows enough to construct the root datum for $\tilde\G$, and its corresponding
involution. For the description of real forms, we use for now the tables
from Helgason \cite{helgason:2001} Table VI in Chapter X.

\subsection{}\label{subsection:weight lattice}{\em Entering a group into the
system~: character lattice.} The practical description of the
lattice $X$ may be done as follows. For each simple factor of $\tilde G$, we
carry out a Smith normal form process to find a basis of the weight lattice
such that suitable multiples of the basis vectors generate the root lattice.
This yields an explicit description of the quotient as a product of
finite groups (in fact it is always cyclic, except for type D in even rank,
where it is a group $\Z_2\times\Z_2$.) Adding the torus factor, we now have
an explicit description of the quotient $X(\tilde G)/Q$. Then $X/Q$ may be any
finite index subgroup of $\tilde X/Q$, stable under involution; one convenient 
way to describe it is
in terms of a set of generators. This is how we do it in the program.

It should be remarked that in this situation, $X$ is a sublattice of 
$X(\tilde G)$ which contains the roots, and $X^\vee$ is a superlattice of
$X^\vee(\tilde G)$, so it obviously contains the coroots. In this setup,
the images of the roots in $X$ and the coroots in $X^\vee$ automatically
constitute a root datum. So it is not hard to get the full root datum of
our real reductive group once we have its weight lattice as a sublattice
of $X(\tilde G)$.

\subsection{}The real form may be described using either the Galois or the
Cartan involution; we will use the Cartan involution $\theta$. As usual, we
denote $\k$ (resp.\ $K$, $\K$) the set of fixed points of $\theta$ in $\g$ 
(resp. $G$, $\G$), and $\p$ the $-1$-eigenspace of $\theta$ in $\g$. Of
course, $K$ is a maximal compact subgroup in $G$. The real
torus we used in our root datum is assumed to be a maximally split
$\theta$-stable torus in $G$; any two such are conjugate under $K$.

The involution induced by $\theta$ on the character lattice of $G$ is just
the negative of the Galois involution. It is entirely described by the
so-called Satake diagram. To get the involution from the Satake diagram,
assume for simplicity that the Lie algebra is absolutely simple (the complex
and torus factors are easy.) Then form a basis of the vector space
$X\otimes_\Z\Q$ as follows~: take a root vector for each black node, and
a weight vector for each white node. Interchange and change signs for
linked white nodes; change sign for unlinked white nodes; do nothing for
black nodes. Then come back to the original basis of the lattice. Actually
this happens already in stage \ref{subsection:lie type}; restricting the
involution to a sublattice is linear algebra.

\subsection{}\label{subsection:torus components}{\em Component groups~: torus
case.} Now we come to what is conceptually the most delicate part, {\em viz.}
the determination of component groups. We start with the torus case.

Let $\T$ be a torus defined over $\R$, $T=\T(\R)$, $\pi_0(T)$ the
component group of $T$. We will see that $T$ decomposes as a direct product
of factors $\U(1)$, $\R^\times$ and $\C^\times$; it follows that $\pi_0$ is
an elementary $2$-group of the form $\Z_2^r$, where $r$ is the number of
factors $\R^\times$. We wish to address the following two problems~: {\em (a)} 
describe $\pi_0(T)$ in terms of the data defining $\T$ (the character lattice 
and the involution) {\em (b)} describe the effect of any homomorphism 
$\f:\T\ra\T'$ on $\pi_0$. 

Let $X=X(\T)$, $X_\Q=X\otimes_\Z\Q$ and let $X_+$ and $X_-$ be the 
intersections of $X$ with the $\pm1$-eigenspaces of $\theta$ in $X_\Q$. Since 
any $\l\in X$ may be written in $X_\Q$ as 
$\frac{1}{2}(\l+\theta(\l))+\frac{1}{2}(\l-\theta(\l))$, it is clear that
we have the inclusions~:
$$
2X\subset X_+\oplus X_-\subset X
$$
Both $X_+$ and $X_-$ have complements in $X$; hence $X_s:=X/X_+$ and 
$X_c:=X/X_-$ are again lattices. They correspond to subtori $\T_s$ and $\T_c$
of $\T$, which we shall call the {\em split part} and the {\em compact part}
of $\T$ (because they correspond respectively to split and compact real forms.)
The natural map $\T_c\times\T_s\ra\T$ is dual to the injection 
$X\ra X_c\oplus X_s$; therefore its kernel (which is the antidiagonal injection
of $\T_c\cap\T_s$ into $\T_c\times\T_s$) is dual to the
elementary $2$-group $(X_c\oplus X_s)/X$.

From this it follows easily that each real torus is isomorphic to a product
of split, compact or complex factors. Indeed, denote $\T(2)$ the subgroup of 
elements of order two in $\T$, and use similar notation for other tori, and for
their groups of real points. Then by a suitable integral base change one
may write $\T_s=\T'_s\times\T''_s$, $\T_c=\T'_c\times\T''_c$, in such a way
that $\T_s\cap\T_c$ is $\T''_s(2)=\T''_c(2)$. Then $\T'_s$ is a split direct
factor in $\T$, $\T'_c$ a compact one, and 
$\T''_s.\T''_c=\T''_s\times\T''_c/(\T''_s\cap\T''_c)$ a complex one.

Using this decomposition, it also follows that the canonical surjection
$\T_s\times\T_c\ra\T$ induces a surjection on the groups of real points;
perhaps there is a more direct argument but I don't see it right now.
Since compact and complex tori have connected groups of real points, the
rank of the component group of $T$ is equal to the rank of the torus $\T'_s$
above; also the rank of the lattice $X_-$ minus the number of complex factors.

It is convenient to consider $\pi_0(T)$ as the quotient group of $T(2)$ by
$T_c(2)$. Notice that when there are complex factors, we have $T(2)\neq\T(2)$,
{\em i.e.,} not all points of order two in $\T$ are real. To study this a
little bit more precisely, denote $V=X/2X$, considered as an $\F_2$-vector
space. The map induced by $\theta$ on $V$ is unipotent; since $T(2)$ is the
kernel of $\theta-1$ on $\T(2)$, its orthogonal is the image of $\theta-1$
in $V$. Denote $V_+$, $V_-$ the images of $X_+$, $X_-$ in $V$. Then it is
not hard to show that $\Im(\theta-1)=V_++V_-$, 
$\Ker(\theta-1)=V_{+-}:=V_+\cap V_-$. So the dual group $T(2)^\vee$ identifies
naturally with $V/V_{+-}$. Since the character lattice of $\T_c$ is by
definition $X/X_-$, it follows that the orthogonal of $\T_c(2)=T_c(2)$ in
$V$ is $V_-$; and its orthogonal in $T(2)^\vee$ is the canonical image
of $V_-$ in $V/V_{+-}$. This is our realization of the dual of the component
group of $T$.

It is now an easy matter (at least in principle---it still requires some
programming) to deduce the map induced at the level of dual component groups
by any homomorphism $\T\ra\T'$.

\subsection{}\label{subsection:components}{\em Component groups~: general case}
Now we explain how to compute the component group of $G=\G(\R)$.  We use the 
following two facts~: {\em (a)} when $\G$ is semisimple and simply connected, 
$G$ is connected; {\em (b)} for any maximally split $\R$-torus $\T$ in $\G$, 
the group $T$ meets all the connected components of $G$ (references ??). 

Already this shows that the map from $\pi_0(T)$ to $\pi_0(G)$ is surjective, 
and hence that $\pi_0(G)$ is an elementary $2$-group. We will take advantage
from this fact to identify the dual group $\pi_0(G)^\vee$ with a subgroup
of our group $\pi_0(YT)^\vee$ determined in \ref{subsection:torus components}.

Now consider a finite covering $\f:\tilde\G\ra\G$. It is clear that this 
induces a surjection $\tilde G^\circ\ra G^\circ$, and since the kernel of $\f$
is necessarily contained in $\tilde T$, we have also a surjection from
$\pi_0(\tilde T\cap\tilde G^\circ)$ to $\pi_0(T\cap G^\circ)$. Passing to
the orthogonals, which are precisely the component groups of $\tilde G$ and
$\G$, we see that $\pi_0^\vee(G)$ is the inverse image in $\pi_0(T)^\vee$
of $\pi_0(\tilde T)^\vee$ by the map $\pi_0(T)^\vee\ra\pi_0(\tilde T)^\vee$
induced by $\f$.

In the case where $\G$ is semisimple, it follows from this and {\em (a)}
above that $\pi_0(G)^\vee$ may be computed by taking $\tilde G$ to be the 
simply connected cover; $\pi_0(G)^\vee$ is then simply the kernel of the
map from $\pi_0(T)^\vee$ to $\pi_0(\tilde T)^\vee$. In the general case,
one may always find a finite cover of the form $\tilde G=\H\times\S$, where
$\H$ is semisimple simply connected, and $\S$ is a torus. The component and
dual component groups of $\tilde G$ are then those of $\S$, and the component
group of $\G$ is computed by the inverse image construction above.

\subsection{}\label{SL(2)}{\em Example~: $\GL(2,\R)$.} Let $\G$ be the split 
$\GL(2)$,
$G=\GL(2,\R)$. Consider the diagonal torus, with its obvious identification
to $(\C^\times)^2$, yielding an identification of $X$ with $\Z^2$. The
involution $\theta$ is identically $-1$, so that of course $\pi_0(T)$ is
$\F_2^2$. We may take $\tilde G=\SL(2)\times\C^\times$. If we identify the
torus of $\SL(2)$ with $\C^\times$ using the first coordinate, the inclusion
map $X\subset\tilde X$ becomes $(x_1,x_2)\ra(x_1-x_2,x_1+x_2)$. Clearly this
is a sublattice of index $2$. The group $\pi_0(\tilde T)$ is also $\F_2^2$,
and if we look at the induced map $\F_2^2\ra\F_2^2$, we see that the first
coordinate of the image of $(a_1,a_2)$ is zero if and only if $a_1=a_2$.
So we find that $G$ has two connected components, as expected, and that
the components of $T$ which lie in the identity component of $G$ are those
with equal signs in both coordinates, also as expected.

\section{The complex Weyl group}\label{section:cweyl}

{\em In this section we will explain our approach to the implementation of the
complex Weyl group. Here the main issue is to choose the representation of
the group elements. Right now my guess is it might be best to come back to
a representation in terms of ``arrays''; of course going from there to a
matrix representation should be available.}

\section{Maximal compact subgroup}\label{section:maxcpct}

{\em Determination of the maximal compact subgroup. The determination of the
topology ($\pi_0$ and $\pi_1$) may be somewhat delicate; also, the lifting
of the component group as a subgroup, which I believe is always true.}

\section{Cartan subgroups}\label{section:cartan}

{\em Determination of the conjugacy classes of Cartan subgroups. This will
probably be done in terms of strongly orthogonal sets of roots. A large
amount of data should be attached to each Cartan~: the involution it defines
on the root datum, the classification of roots, the corresponding Levi
subgroup as a real reductive group, complete with the topological data, ...}

\section{Weyl groups}\label{section:rweyl}

{\em Determination of the real Weyl group attached to a given Cartan subgroup.
I'm not sure I understand all the details here yet.}

\bibliographystyle{plain}
\bibliography{structure}

\end{document}

\section{The complex Weyl group}\label{section:cweyl}

{\em In this section we will explain our approach to the implementation of the
complex Weyl group. Here the main issue is to choose the representation of
the group elements. Right now my guess is it might be best to come back to
a representation in terms of ``arrays''; of course going from there to a
matrix representation should be available.}

\section{Maximal compact subgroup}\label{section:maxcpct}

{\em Determination of the maximal compact subgroup. The determination of the
topology ($\pi_0$ and $\pi_1$) may be somewhat delicate; also, the lifting
of the component group as a subgroup, which I believe is always true.}

\section{Cartan subgroups}\label{section:cartan}

{\em Determination of the conjugacy classes of Cartan subgroups. This will
probably be done in terms of strongly orthogonal sets of roots. A large
amount of data should be attached to each Cartan~: the involution it defines
on the root datum, the classification of roots, the corresponding Levi
subgroup as a real reductive group, complete with the topological data, ...}

\section{Weyl groups}\label{section:rweyl}

{\em Determination of the real Weyl group attached to a given Cartan subgroup.
I'm not sure I understand all the details here yet.}

Denote $V=X/2X$, considered as a vector
space over the two-element field $\F_2$; then $V$ is also the dual group of 
$\T(2)$. The map induced by $\theta$ on $V$ is unipotent, with 
$(\theta-1)^2=0$. Denote $V_+$, $V_-$ the images of $X_+$, $X_-$ in $V$, 
and $V_0=V_+\cap V_-$. Then it is not hard to see that we have 
$\Im(\theta-1)=V_0$, $\Ker(\theta-1)=V_+\cap V_-$. Let $r$ be the dimension
of $V_0$, and $p+r$, $q+r$, the dimensions of $V_+$ and $V_-$. Then we may
choose bases $v'_1,\ldots,v'_{p+r}$ of $X_+$, $v''_1,\ldots,v''_{q+r}$ of
$X_-$, such that $v'_{p+j}=v''_{q+j}\mod2$ for all $1\leq j\leq r$. Then the
claim is that $v'_1,\ldots,v'_p$, $v''_1,\ldots,v''_q$, and the 
$\frac{1}{2}(v'_{p+j}\pm v''_{q+j})$ form a basis of the lattice $X$. It is
enough to show that these elements generate $X$; and since they clearly
generate all elements in $X_+\oplus X_-$, it suffices to prove that their
images in $V$ generate $V$. But from the uniqueness of the expression of
an element of $X_\Q$ in the basis of the $v'_i$ and $v''_j$, it follows
that no $\Z$-linear combination of the $\frac{1}{2}v'_{p+j}+v''_{q+j}$ with 
coefficients not all even can lie in $X_+\oplus X_-$. Hence the images of
these vectors in $V$ are independent modulo $V_++V_-$, and therefore they,
together with $V_+$ and $V_-$, generate $V$. Now $X$ is the direct sum
of the lattices $\Z v'_j$, $1\leq j\leq p$, $\Z v''_j$, $1\leq j\leq q$,
and $\frac{1}{2}\Z(v'_{p+j}+v''_{q+j})\oplus\frac{1}{2}\Z(v'_{p+j}-v''_{q+j})$,
which correspond respectively to compact, split and complex factors of our
torus (notice that the involution exchanges the two vectors 
$\frac{1}{2}(v'_{p+j}\pm v''_{q+j})$.)

